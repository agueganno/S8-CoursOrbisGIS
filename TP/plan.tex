\documentclass[a4paper]{article}
% generated by Docutils <http://docutils.sourceforge.net/>
\usepackage{fixltx2e} % LaTeX patches, \textsubscript
\usepackage{cmap} % fix search and cut-and-paste in Acrobat
\usepackage{ifthen}
\usepackage[T1]{fontenc}
\usepackage[utf8]{inputenc}

%%% Custom LaTeX preamble
% PDF Standard Fonts
\usepackage{mathptmx} % Times
\usepackage[scaled=.90]{helvet}
\usepackage{courier}

%%% User specified packages and stylesheets

%%% Fallback definitions for Docutils-specific commands

% hyperlinks:
\ifthenelse{\isundefined{\hypersetup}}{
  \usepackage[colorlinks=true,linkcolor=blue,urlcolor=blue]{hyperref}
  \urlstyle{same} % normal text font (alternatives: tt, rm, sf)
}{}
\hypersetup{
  pdftitle={Création d'une fonction table pour gdmsql},
}

%%% Title Data
\title{\phantomsection%
  Création d'une fonction table pour gdmsql%
  \label{creation-d-une-fonction-table-pour-gdmsql}}
\author{}
\date{}

%%% Body
\begin{document}
\maketitle


%___________________________________________________________________________

\section*{\phantomsection%
  Pour gagner du temps...%
  \addcontentsline{toc}{section}{Pour gagner du temps...}%
  \label{pour-gagner-du-temps}%
}

Afin de pallier aux déboires rencontrés la semaine dernière, vous allez
récupérer les sources directement sur

\url{http://brehat.ec-nantes.fr/S8/S8Li.tar.gz}

Cette archive contient les sources du logiciel, tronquée des sources et
ressources de test. Le tout représente un peu moins de 100Mo de données.

Étant donné qu'il s'agit du dernier TP, vous pouvez mettre vos travaux dans
un dossier temporaire (/tmp) pour être sûr d'avoir suffisamment d'espace.

Noter que, quoiqu'il arrive, entre 70 et 100 Mo seront nécessaires à Maven pour
récupérer ses modules et les dépendances d'OrbisGIS, et que cet espace sera pris
sur votre espace personnel. Vous ne pouvez pas (du moins pas simplement)
rediriger cette charge vers /tmp.

Une fois le téléchargement et l'extraction réalisées, vous pouvez l'ouvrir
directement grâce à NetBeans. Et commencer à coder, enfin...


%___________________________________________________________________________

\section*{\phantomsection%
  Pour gagner de l'espace%
  \addcontentsline{toc}{section}{Pour gagner de l'espace}%
  \label{pour-gagner-de-l-espace}%
}

Étant donné le checkout avorté du dernier TP, de l'espace disque est utilisé
inutilement dans votre espace utilisateur. Comme vous allez récupérer les
archivées, vous pouvez supprimer ces données devenues inutiles.

Par défaut, les codes sources récupérés avec SVN ont été placés dans votre
dossier personnel, dans un dossier qui s'appelle NetBeansProjects. Faites le
ménage dans ce dossier, et votre espace personnel respirera de nouveau...


%___________________________________________________________________________

\section*{\phantomsection%
  Installation des sources d'OrbisGIS%
  \addcontentsline{toc}{section}{Installation des sources d'OrbisGIS}%
  \label{installation-des-sources-d-orbisgis}%
}

L'installation des sourcse du logiciel va se faire en deux parties :
%
\begin{itemize}

\item Nous allons utiliser l'environnement de développement intégré NetBeans. Pour
le lancer, vous pouvez ouvrir une console et utiliser la commande (à
adapter) :

\end{itemize}
%
\begin{quote}{\ttfamily \raggedright \noindent
/opt/netbeans-x/bin/netbeans
}
\end{quote}
%
\begin{itemize}

\item La procédure d'installation des sources avec NetBeans peut être trouvée à
l'adresse \url{http://trac.orbisgis.org/t/wiki/devs/OrbisGIS_src_netbeans} . Une
fois au bout des instructions, vous serez en mesure de commencer à coder...

\end{itemize}


%___________________________________________________________________________

\section*{\phantomsection%
  Objectif du TP%
  \addcontentsline{toc}{section}{Objectif du TP}%
  \label{objectif-du-tp}%
}

Vous allez ajouter une fonction table à la grammaire de gdmsql. Vous pourrez
vous appuyer sur le cours précédent (disponible sur le serveur pédagogique),
ainsi que sur toutes les ressources disponibles sur Internet.


%___________________________________________________________________________

\section*{\phantomsection%
  Définition de la fonction%
  \addcontentsline{toc}{section}{Définition de la fonction}%
  \label{definition-de-la-fonction}%
}


%___________________________________________________________________________

\subsection*{\phantomsection%
  Nom de la fonction%
  \addcontentsline{toc}{subsection}{Nom de la fonction}%
  \label{nom-de-la-fonction}%
}

La fonction s'appellera ST\_BufferedMean.


%___________________________________________________________________________

\subsection*{\phantomsection%
  Entrée de la fonction%
  \addcontentsline{toc}{subsection}{Entrée de la fonction}%
  \label{entree-de-la-fonction}%
}
%
\begin{itemize}

\item Une table contenant deux colonnes, dont l'une contient des objets de type
Point, et l'autre des valeurs numériques.

\item Une valeur de type Double

\item Le nom de la colonne numérique

\item Un booléen optionnel

\end{itemize}


%___________________________________________________________________________

\subsection*{\phantomsection%
  Sortie de la fonction%
  \addcontentsline{toc}{subsection}{Sortie de la fonction}%
  \label{sortie-de-la-fonction}%
}

Une table contenant une colonne de type Point et deux colonnes de type
numérique.


%___________________________________________________________________________

\subsection*{\phantomsection%
  Traitement%
  \addcontentsline{toc}{subsection}{Traitement}%
  \label{traitement}%
}

Pour chacun des points donnés en entrée, vous devrez :
%
\begin{itemize}

\item Trouver tous les autres points de la table dont la distance au point courant
est inférieure au double passé en paramètre de la fonction

\item Calculer la valeur moyenne du champ numérique associé à chacun de ces points

\item Ajouter une ligne à la DataSource de sortie contenant le point, la valeur
associée au point et la moyenne calculée précédemment

\end{itemize}

Le booléen passé en paramètre permettra de décider si la valeur numérique
associée au point courant doit être incluse dans le calcul de la moyenne.


%___________________________________________________________________________

\section*{\phantomsection%
  Contraintes%
  \addcontentsline{toc}{section}{Contraintes}%
  \label{contraintes}%
}
%
\begin{itemize}

\item La fonction sera placée dans un package dédié.

\item La fonction devra s'exécuter avec succès en moins de cinq minutes sur les
données de test.

\end{itemize}


%___________________________________________________________________________

\section*{\phantomsection%
  Ressources%
  \addcontentsline{toc}{section}{Ressources}%
  \label{ressources}%
}

Quelques javadoc intéressantes :
%
\begin{itemize}

\item Java \url{http://docs.oracle.com/javase/6/docs/api/}

\item JTS \url{http://tsusiatsoftware.net/jts/javadoc/index.html}

\item OrbisGIS (par module) : \url{http://javadoc.orbisgis.org/}

\end{itemize}

Ces documentations devraient répondre à la plupart des questions soulevées par
le sujet. Elles viennent en complément du cours précédent ce TP.

Pour la gestion des index :
%
\begin{itemize}

\item \url{http://trac.orbisgis.org/t/wiki/devs/GDMS2/gdms_index}

\end{itemize}


%___________________________________________________________________________

\section*{\phantomsection%
  Données%
  \addcontentsline{toc}{section}{Données}%
  \label{donnees}%
}

Les données de test peuvent être récupérées à l'adresse :

\url{http://brehat.ec-nantes.fr/S8/}

Le fichier est nommé receivers\_dBA.gdms. Il contient toutes les informations
nécessaires à l'exécution de la fonction désirée...

\end{document}
